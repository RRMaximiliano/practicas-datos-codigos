%!TEX program = pdflatex
% Full chain: pdflatex -> bibtex -> pdflatex -> pdflatex
\documentclass[11pt,en]{elegantpaper}

\title{Investigación reproducible \\ Buenas prácticas para el manejo de datos y códigos\footnote{Traducción al español a cargo de: Sebastián Hernández y Rony Rodríguez-Ramírez.}}
\author{Harrison Diamond Pollock \and Erica Chuang \and Stephanie Wykstra}
\date{\today}


\begin{document}
%----------------------------------------------------------------------------------------
\maketitle
\tableofcontents
%----------------------------------------------------------------------------------------
\newpage 
%----------------------------------------------------------------------------------------
\section{Introducción}
Las revistas, los financiadores de investigación y los grupos de investigación como Innovations for Poverty Action reconocen cada vez más el valor de la transparencia en la investigación. La transparencia de la investigación incluye el registro previo de estudios y el intercambio de materiales como datos y códigos para permitir que otros vuelvan a analizar los resultados informados.

La gestión adecuada de datos y códigos durante un proyecto es esencial para la transparencia después de la finalización de un proyecto. También son importantes para uso interno, ya que los proyectos a menudo se ejecutan durante varios años, con varios miembros del personal trabajando en ellos de forma secuencial.

Esta guía describe las mejores prácticas en el manejor de datos y códigos. El alcance de la guía es cubrir los principios de organización y documentación de materiales en todos los pasos del ciclo de vida de un proyecto con el objetivo de hacer que la investigación sea reproducible. La guía no cubre las mejores prácticas para diseñar encuestas, limpiar datos o realizar análisis de datos. En cada sección, explicamos el "qué", el "por qué" y el "cómo" de cada práctica recomendada.

Para comentarios/preguntas, comuníquese con \href{mailto:researchsupport@poverty-action.org}{researchsupport@poverty-action.org}.
%----------------------------------------------------------------------------------------
\section{Carpetas y estructura de archivos}
%----------------------------------------------------------------------------------------

%----------------------------------------------------------------------------------------
\section{Mejores prácticas para códigos}
%----------------------------------------------------------------------------------------

%----------------------------------------------------------------------------------------
\subsection{Nombrar y etiquetar variables}
%----------------------------------------------------------------------------------------

%----------------------------------------------------------------------------------------
\subsection{Manejo de valores perdidos}
%----------------------------------------------------------------------------------------

%----------------------------------------------------------------------------------------
\subsection{Escribiendo do-files}
%----------------------------------------------------------------------------------------

%----------------------------------------------------------------------------------------
\subsection{Usando referencias relativas}
%----------------------------------------------------------------------------------------

%----------------------------------------------------------------------------------------
\section{Mantener un buen código y documentación de datos}
%----------------------------------------------------------------------------------------

%----------------------------------------------------------------------------------------
\section{PII y mantener seguros los datos y el código}
%----------------------------------------------------------------------------------------
\end{document}
